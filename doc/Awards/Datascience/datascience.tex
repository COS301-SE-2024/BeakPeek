\documentclass{article}

\usepackage{listings}

\title{

\\
{Installation Manual}
}

\author{Millenium}

\begin{document}

\tableofcontents

\newpage


\section{Introduction}


BeakPeek was developed with a backend and frontend and will there fore have two
seperate sub sections for installation for the frontend and backend installation
\section{Algorithms}
\subsection{Data Science behind the HeatMap}

The HeatMap feature processes and visualizes a vast dataset, encompassing over \textbf{16,000 pentads} (geographic regions) and more than \textbf{1.2 million bird entries}. To manage such large-scale data, the app utilizes various algorithms to optimize data handling, visualization, and performance.

\subsubsection{Dynamic Pentad Data Loading}
Given the scale of the data, the \texttt{loadPentadData} function dynamically loads bird sighting data only as needed. Instead of attempting to load all 16,000 pentads and millions of bird entries simultaneously, the algorithm fetches data relevant to the user's selections, such as species and month, and updates the map accordingly.

\textbf{Why it's effective:}
\begin{itemize}
    \item \textbf{Data Efficiency:} By fetching only necessary data based on user input, the app minimizes memory usage and processing time, allowing it to handle a database with over 1.2 million entries efficiently.
    \item \textbf{Batch Processing:} The large dataset is processed in small increments (batches), preventing performance degradation and ensuring smooth interaction even with vast amounts of data.
\end{itemize}

\subsubsection{Data Visualization with Color Coding}
With a database of this size, data visualization is key to transforming raw data into actionable insights. The \texttt{getColorForReportingRate} function assigns colors to polygons based on the reporting rate (i.e., bird sighting percentages in each pentad). This color coding makes it easier to interpret patterns across thousands of pentads.

\textbf{Why it's effective:}
\begin{itemize}
    \item \textbf{Data-Driven Visualization:} Transforming over a million bird entries into an intuitive color-coded map allows users to quickly understand the distribution and frequency of sightings.
    \item \textbf{Scalable Visualization:} As users zoom in or filter the data, the map dynamically recolors the polygons to reflect updated sighting patterns, providing immediate visual feedback.
\end{itemize}

\subsubsection{Interactive Map Filtering}
With 16,000 pentads in the dataset, filtering is essential for managing data. The app allows users to filter bird sighting data by month, significantly reducing the dataset to a manageable size while maintaining interactivity.

\textbf{Why it's effective:}
\begin{itemize}
    \item \textbf{Time-Based Filtering:} Birds migrate seasonally, and sightings vary by month. The app allows users to focus on specific months, making the data more relevant and easier to interpret.
    \item \textbf{Interactive Scalability:} Filtering by time and species reduces the complexity of rendering the data, allowing the system to handle large datasets efficiently and update the map in real-time.
\end{itemize}

\subsubsection{Efficient Data Rendering}
Rendering polygons for 16,000 pentads and over a million bird entries is computationally intensive. The app uses the Google Map widget to render polygons dynamically, processing bird sighting data in small batches to avoid performance bottlenecks.

\textbf{Why it's effective:}
\begin{itemize}
    \item \textbf{Real-Time Data Processing:} The app only processes and renders polygons relevant to the current view or filter, keeping the performance optimized.
    \item \textbf{Performance Optimization:} By processing and rendering polygons in batches, the app remains responsive, even when dealing with large datasets.
\end{itemize}

\subsubsection{Endangered Species Alerts}
The app also tracks bird population data, providing real-time alerts when a species is endangered. The population size is checked against thresholds, and users are alerted if the species they are viewing is at risk.

\textbf{Why it's effective:}
\begin{itemize}
    \item \textbf{Real-Time Alerts:} The app dynamically analyzes bird population data, providing conservation insights and warnings if a species is endangered.
    \item \textbf{Data-Driven Decision Making:} Based on the vast dataset, the app can make real-time decisions on whether to show alerts about endangered species, enhancing both the educational and conservation aspects of the feature.
\end{itemize}

\subsubsection{Conclusion}
These algorithms enable the HeatMap feature to efficiently manage and visualize over 1.2 million bird sightings across 16,000 pentads. By dynamically loading, filtering, and rendering data, the app delivers a responsive and scalable user experience, leveraging data science to provide actionable insights on bird populations and conservation.

\subsection{Data Science behind the BirdMap}

The BirdMap feature handles a large dataset consisting of \textbf{16,000 pentads} (geographic regions) and over \textbf{1.2 million bird entries}. The following algorithms ensure efficient data management, visualization, and interactivity on the map.

\subsubsection{Dynamic Location and KML Data Loading}
The \texttt{\_getCurrentLocation} function dynamically centers the map based on the user's current location, while the \texttt{\_loadKmlData} function loads KML data that represents polygon shapes for provinces. This approach allows the app to manage large datasets and visualize geographic information in real-time.

\textbf{Why it's effective:}
\begin{itemize}
    \item \textbf{Data Efficiency:} By dynamically loading polygon data for specific provinces, the app minimizes the amount of data that needs to be processed, ensuring that only the relevant information is displayed.
    \item \textbf{Scalable Processing:} Instead of attempting to load all 16,000 pentads or 1.2 million bird entries at once, the app fetches data on demand, optimizing performance and memory usage.
\end{itemize}

\subsubsection{Data Filtering by Province and Month}
The map provides filtering options based on province and month, allowing users to narrow down the data to specific geographic areas and time frames. This significantly reduces the amount of data the app needs to process and display.

\textbf{Why it's effective:}
\begin{itemize}
    \item \textbf{Interactive Data Reduction:} Filtering helps manage the large dataset by focusing on a specific province or month, making the data more relevant and easier to handle.
    \item \textbf{Efficient Querying:} By querying the backend for bird sightings based on the selected filters, the app reduces the need to handle unnecessary data, improving both performance and responsiveness.
\end{itemize}

\subsubsection{Polygon-Based Data Visualization}
The polygons representing each pentad are color-coded based on bird sighting reporting rates. This visual approach transforms the large dataset into an easy-to-understand map, where users can quickly identify areas with high or low bird sightings.

\textbf{Why it's effective:}
\begin{itemize}
    \item \textbf{Visual Representation of Data:} Color-coded polygons provide an intuitive way to understand bird sighting distribution across thousands of pentads, transforming over 1.2 million entries into a simple visual format.
    \item \textbf{Scalable Visualization:} As the user zooms in or filters the data, polygons are updated in real-time, ensuring the map stays relevant and responsive.
\end{itemize}

\subsubsection{Real-Time Map Interaction}
The app leverages Google Maps for real-time interaction, allowing users to zoom, pan, and click on polygons to get more information about bird sightings in specific regions. Each polygon represents a pentad, and tapping on it provides detailed data about birds in that area.

\textbf{Why it's effective:}
\begin{itemize}
    \item \textbf{Real-Time Data Updates:} Users can interact with the map to view specific bird data, with the polygons and information updating in real-time based on user actions.
    \item \textbf{Efficient Event Handling:} The app only updates the relevant portions of the map (such as polygons) based on user input, maintaining high performance even with large datasets.
\end{itemize}

\subsubsection{Conclusion}
The BirdMap feature efficiently handles over 1.2 million bird entries and 16,000 pentads using dynamic data loading, filtering, and polygon-based visualization. These algorithms ensure that the app remains responsive, interactive, and scalable, providing users with real-time insights into bird populations and sightings.

\section{Explanation of Algorithms in LifeList}
The LifeList's algorithms are geared towards performance ensuring that the user always has access to the data with minimal delays, ensuring that it the App remains efficient and responsive.The SQLite database that is used has also been designed to be scalable whilst still maintaining efficiency. 

\subsection{SQLite Tables}
The LifeList database is comprised of 3 tables Birds, allBirds and Provinces

\subsubsection{AllBirds Table}
The allBirds table stores all the data for the 800+ birds in South Africa.Images are also store as a Blob of base64. The user also has the ability to request for this table to be updated giving the latest data when they require it.

\textbf{Why it's effective:}
\begin{itemize}
    \item \textbf{Offline access:} By keeping this table the user is able to access all information about all the birds in South Africa at anytime regardless of internet access.
    \item \textbf{Better response time:} Having no need to constantly make request to the online API means that the information is always readily available and able to be displayed.
    \item \textbf{Lower storage requirements:} By storing images as a string of base64 we drastically decease the size of the database 
    \item \textbf{Redundancy:} All birds also stores the online URL for the images this means that should the image fail to load the online image is retrieved and stored in the table for next usage.
\end{itemize}

\subsubsection{Bird Table }
The \texttt{Bird} table stores rudimentary bird data for birds that have been seen.

\textbf{Why it's effective:}
\begin{itemize}
    \item \textbf{Saves storage:} By keeping this table small we minimize storage requirements.
    \item \textbf{Seamless Integration:} As the \texttt{Bird} stores minimal information it make it easier to upload to the users online account allowing for the transfer of data to be quicker and more cost effective.
\end{itemize}

\subsubsection{Provinces Table }
The \texttt{Provinces} table stores the Bird is and a list of provinces that the bird is found in.

\textbf{Why it's effective:}
\begin{itemize}
    \item \textbf{Efficiency :} This table allows for 1 query to be made to find number of birds in a province and which birds are found in which provinces.
    \item \textbf{Storage Efficiency:} This allows the \texttt{AllBirds} table to be made smaller and faster.
\end{itemize}

\subsubsection{Online Profile Update}
The\texttt{\ fetchUserLifelistString} retrieves the necessary data from the \texttt{\ Birds} table that is need for online profile storage.

\textbf{Why it's effective:}
\begin{itemize}
    \item \textbf{Efficiency :} This allows for efficient retrieval of the life list to be stored online .
    \item \textbf{Storage Efficiency:} This allows the online storage to be smaller and more manageable.
\end{itemize}

\subsubsection{Duplicate inserting}
The\texttt{\ isDuplicate} checks if the bird to be inserted is in the life list already.

\textbf{Why it's effective:}
\begin{itemize}
    \item \textbf{Efficiency :} This prevents duplicate data from even being attempted to be inserted and it allows the front end to identify if a bird is in the Life List without being queried.
\end{itemize}

\subsubsection{Bird Provinces}
The\texttt{\ getBirdProvinces} return all the provices that a bird is in.

\textbf{Why it's effective:}
\begin{itemize}
    \item \textbf{Efficiency :} Allow for 1 query that retrieves all provinces that a bird is seen in.
\end{itemize}

\subsubsection{Achievement Progress}
The\texttt{\ precentLifeListBirds} returns the percentage of the current achievement. By querying \texttt{\ Provinces} and \texttt{\ Birds} tables we are able to calculate the percentage of the achievements.This query is called upon inserting a bird into life list and the value is stored in the user model.

\textbf{Why it's effective:}
\begin{itemize}
    \item \textbf{Efficiency :} Allow for 1 query that retrieves the current percentage.
    \item  \textbf{Seamless Integration :} By storing this data in the user model it is always available when need so loading time is minimized.
\end{itemize}

\subsubsection{Overall Design}
These tables and algorithms ensure that the LifeList as well as all bird information is available at all times. By loading the LifeList as the app starts we ensure that the data is always readily available to the user anytime the user needs it. These algorithms are also geared to minimize loading times and save storage space by breaking the data down into smaller and more manageable tables. 
\end{document}
