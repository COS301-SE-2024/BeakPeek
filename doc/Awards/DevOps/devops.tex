\documentclass[a4paper]{article}
\usepackage{graphicx}
\usepackage{xcolor}
\usepackage[margin=1in]{geometry}
\usepackage[utf8]{inputenc}
\usepackage[absolute]{textpos}
\usepackage{listings}

\title{

\\
{DevOps}
}

\author{Millennium}

\begin{document}

\begin{titlepage}

\begin{center}

{\Huge \bf BeakPeek}\\[10pt]
\rule{\textwidth}{1pt}\\[20pt]
{\bf \huge \sc Millennium} \\[10pt]
{\huge University Of Pretoria} \\[15pt]
\includegraphics[width=7cm]{../../../res/MillenniumLogo.jpeg}\\[15pt]
{\huge Algorithmic Innovation Award}\\[60pt]
\rule{8cm}{1pt}\\[5pt]
{\bf \huge \sc In Partnership With}\\[15pt]
{\huge \bf AgileBridge}\\[5pt]
\rule{8cm}{1pt}\\[50pt]

\begin{minipage}[t]{10cm}
	{\Large{\bf Group:\\ 22}}
\end{minipage}\hfill\begin{minipage}[t]{5cm}\raggedleft
	{\Large{\bf Mentor: \\Ayaaz\\ }}
\end{minipage} \\[90pt]
{\Large 2024/10/18} \\ [5pt]

\rule{\textwidth}{1pt}\\[10pt]

\end{center}

\vfill

\end{titlepage}

\newpage

\section{Introduction}

We as Millennium have from the start of the capstone project placed a high 
level of importance on DevOps which was managed by and maintained by Ivan Horak
who has been our DevOps engineer.\newline

Seeing that the scale of this project is so large we feel it is important that we 
give a summary of what BeakPeek is and how we have structured the repository and 
all of the different parts of how BeakPeek functions.

\subsection{What is BeakPeek}
BeakPeek is a bird watching app that intends to making birding as easy as fly.\newline

Our core functional requirements are:
\begin{itemize}
    \item Data ingress and transformation
    \item Usage of the data from SABAP (South African Bird Atlas Project) for
        \begin{itemize}
            \item Heat map generation by pentad (A 5 minute by 5 minute area)
            \item Bird population estimation
            \item Bird quiz generation
        \end{itemize}
    \item User management
    \item Bird Achievements
    \item Lifelists (list of birds that a user has seen)
\end{itemize}

\subsection{Project/Repository structure}

Our project has been split into a couple common directories. 
Those being:
\begin{itemize}
    \item doc: For documentation
    \item res: For resources
    \item scripts: For bash and SQL scripts
    \item beakpeek: For our flutter frontend
    \item dotnet: For our two .Net backends
        \begin{itemize}
            \item BirdApi: For the bird API 
            \item UserApi: For our user portal and management API 
        \end{itemize}
\end{itemize}

\subsection{What we use?}
The tools that we have used are:
\begin{itemize}
    \item Azure to host the bird API and User API
    \item .Net entity framework core for both API's
    \item .Net identity for the User portal and API
    \item Flutter for mobile app development
    \item MSSQL for both databases also on Azure
\end{itemize}

\subsection{Most Important things to note for DevOps}

\begin{itemize}
    \item We use Azure to host our 2 .Net backends/API's
    \item We use MSSQL for our database
    \item We have an Android release on Google Play
    \item We use GitHub actions for our CI/CD Workflows
    \item We get our data from SABAP
\end{itemize}

\section{Non-Functional Requirements}

We as Millennium set out for ourselve 3 main Non-Functional requirements that we
thought would be the most important and applicable to BeakPeek and would best show 
our wide range of skills and show our abilities as software engineers.

\subsection{Performance}
One of the most important skills as software engineers is the ability to make performant
and scalable solutions. \newline

\subsubsection{Load Testing}
We use Azure Load Testing to simulate 50 users constantly signing up, logging in, 
getting their profile, updating their profile and then finally deleting their profile

To show that we can make highly performant and highly scalable solution we have 
used a load test which is hosted on azure in conjunction with our api's. \newline

To ensure that our users are never unable to login and manage their profiles we 
used a load test to ensure that even if and when we have high volumes of users we 
never drop a request. This is done by simulating 50 users simultaneously and 
constantly signing up, logging in, getting their profile, updating their profile
and finally deleting their account. This is done over 10 minutes which gives us
approximately 2.49 thousand requests with an error percentage of 0.36\%. This 
shows that our User API can handle a high qauntity of users easily.


\subsubsection{Statistics}
insert image here

\subsection{Security}
To ensure our users security and ensure that their data stays secure and that 
only users who are authenticated are allowed to login and edit their data we 
have used Json Web Tokens which are an easy to transport and easy to use authentication
token as it ensures that only users who have been issued a token from our user Api
as anyone could decrypt a JWT but only JWTs issued by our user api are valid as they 
are validated against a secret key on our user API.

\subsubsection{User Portal/Api}

We use JWT (Json Web Tokens) to authenticate our users when they login.

\subsubsection{Usability}


\newpage

\section{CI/CD tools and pipelines}

\subsection{Tools}

\subsubsection{Fastlane}
We have used Fastlane to automate the building and release of the flutter android frontend onto the 
Google play store.

\subsubsection{JMeter}
JMeter

\subsubsection{Codecov}
To ensure that our the anything we make is up to the proper standards and that 
there are no unintended consequences of any new features we add we have employed
the use of Codecov to be able to track our code coverage across our project and 
across the different sections of the project.\newline

We have split up our code coverage tracking into two primary sections

\begin{itemize}
    \item The Flutter front-end
    \item The .Net backend
\end{itemize}

To ensure that our code is properly tested and up to industry standards we have
aimed for atleast a 75\%  line coverage for each different scope of our project.
\newline
In this regard we habe managed to maintain a 91\% code coverage for the BirdApi.

\subsection{Workflows}

\subsubsection{Discord Notify}



\subsection{Pipelines}

\newpage
\section{Observability Principles}

\subsection{Dashboads}
\subsection{Azure Monitoring}
\subsection{GitHub Actions}
We monitor with GitHub Actions

\newpage
\section{Operational Practices}
We have methods of recovering from data loss
\subsection{Data Ingress}
Github action and script for downloading and importing birds from sabab
\subsection{Azure Georedundency}
Azure offers Georedundency

\end{document}
